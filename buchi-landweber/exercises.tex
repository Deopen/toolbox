% !TEX root = ../main.tex

\exercise{zad:quasipol-reach}{
We say that a game is \emph{finite} if it  has no infinite plays, i.e.~every play eventually reaches a dead end.
Prove that every finite game is determined, i.e.~exactly one of the players has a winning strategy.
}
{

}

\exercise{zad:02-02}{
Show that  reachability games played on  finite game graphs can be solved in time proportional to the number of edges.
}
{
Assume that the objective of player $0$ is visiting an edge from a designated subset $X$.  The idea is to compute  the attractors as defined in  the proof of Theorem~\ref{thm:parity-memoryless}.

At each step $i \in \set{1,2,\ldots}$ of the algorithm, we maintain a set of edges $X_i$ which is stored by a data structure which associates to each position  the following data:
\begin{enumerate}
	\item for each position the number of outgoing edges in $X_i$;
	\item for each position, the list of all  incoming edges that are not  in $X_i$.
\end{enumerate}
  Initially, $X_1$ is set to be the edges from the winning condition, i.e.~the edges which player $0$ needs to visit in order to win.  In the $i$-th step, with $i \in \set{1,2,\ldots}$ the algorithm inspects each position $v$ which had an outgoing edge added in the previous step, i.e.~there 
was an edge leaving $v$ that was in $X_i$ but not in $X_{i-1}$, with $X_0$ being assumed to be empty.   Depending on the owner of position $v$, do the following:
\begin{itemize}
	\item \emph{Player $0$ owns $v$.} If there is  at least one outgoing edge from $v$ that belongs to $X_i$ (which is necessarily the case, since an edge was added), then add all edges that enter $v$ to the set $X_{i+1}$.
	 	\item \emph{Player $0$ owns $v$.} If some  edge leaving $v$ is still not in  $X_i$, then do nothing. Otherwise, if all edges leaving $v$ are in $X_i$, then add all edges that enter $v$ to the set $X_{i+1}$.
\end{itemize}
Using the data structures, we can make sure that a constant number of operations is executed when adding an edge to $X_i$. The algorithm stops when $X_i$ stops growing. It accepts if the initial position has (some / all) outgoing edges in the stable set $X_i$, depending on the owner of the initial position. Otherwise, the algorithm rejects.
}



\exercise{zad:02-07}{
Show  one player parity games can be solved in PTIME.
}
{
This amounts to checking whether there is a reachable cycle with the largest
rank of the given parity.
}



\exercise{zad:02-08}{
Show that solving parity games is in $\textup{NP} \cap \textup{coNP}$.
}
{
We will use the fact that in parity games winning strategies can chosen to be positional.
To show presence in \textup{NP} we do the following.
For a vertex in which player $i$ wins we guess a positional strategy for player $i$ (an edge
from every $i$-vertex, so a polynomial object). To check that it is indeed winning for $i$ we have to choose
that in a graph which appeared after fixing a strategy for $i$ the player $i$ indeed wins. It is enough to check
that there is no reachable loop in which the smallest rank has parity $1-i$, which can be easily done in polynomial
time.

In order to show presence in \textup{coNP} we have to show that the complement is in \textup{NP}.
It is easy, as the complement is existence of winning strategy for player $1-i$ and since the game is almost symmetric
this is also in \textup{NP}.
}



\exercise{zad:02-09}{
Consider the following game on a finite game graph $V$ together with function $\rank: V \to \N$.
At every moment of the play, the owner of the current vertex chooses a next vertex among current vertex successors.
This continues until some vertex repeats on the play, i.e. till the first loop is closed. Then depending on the parity
of the smallest rank on the loop the winning player is determined.
Prove that player $i$ in the described game wins iff player $i$ wins in the parity game on the same arena.
}
{
Assume that $i$ wins in the parity game. Then we has a winning strategy which is positional.
Let $i$ play the same strategy in the loop-game. Consider the first closed loop. If the smallest
rank on this loop would have parity $1-i$ it would mean that there exists a strategy of $1-i$
in the parity game, which forces the play to go through that loop all the time and the result would be
that player $i$ would lose. So the smallest rank on the loop has the parity $i$, so $i$ also has a winning strategy
in the loop game.
If $1-i$ wins in the parity game then in the same way we show that $1-i$ wins the loop-game. So indeed these
two games are equivalent.
}



\exercise{zad:02-10}{
Are Muller games  positionally determined?
}
{
There are many ways to show that they are not positionally determined.
One way is to set $\F = \{q_0, q_1\}$ and put a graph with transitions:
$p \trans{} q_0$, $p \trans{} q_1$, $q_0 \trans{} p$ and $q_1 \trans{} p$, where the vertex $p$ is owned by the player,
who wants to have $\inf(\rho) \in \F$. This player has to change his decision from time to time (go sometimes to $q_0$
and sometimes to $q_1$) in order to win.

}



\exercise{zad:02-11}{
Show that B\"{u}chi games are positionally determined without direct use of the same result for parity games.
}
{

}




\exercise{zad:02-12}{
Show that the winning condition  Muller games is a  Borel set, and therefore Muller games are determined by  Martin's theorem. (Most of this problem is looking up what Borel sets and Martin's theorem are.) 
}
{

}



\exercise{zad:02-14}{
Show that Muller games on finite arenas are not positionally determined.
}
{

}




\exercise{zad:02-13}{
Construct an infinite game played on a finite game graph, in which  player 0 has a winning strategy, but not a winning  finite memory strategy.
\emph{Remark:} Notice that by B\"{u}chi-Landweber theorem the winning condition in that
game cannot be $\omega$-regular.
}
{
The same game graph as in exercise~\ref{zad:02-14} works: $p \trans{} q_0$, $p \trans{} q_1$, $q_0 \trans{} p$
and $q_1 \trans{} p$, where winning condition for the player, which owns node $p$ could be that indices of the sequence
of $q_i$ states (so zeros and ones) form some not ultimately periodic word in $\{0, 1\}^\omega$.
For example a binary representation of number $\Pi$. Then it is clear that player owning $p$ can win,
but he needs an infinite memory.
}


\exercise{zad:02-03}{
Consider the following riddle. There are infinitely many dwarfs (countably many). Every dwarf is given a hat, which is either red or green.
Every dwarf sees the color of every hat beside his own one. Every dwarf is supposed to tell what is the color of his hat, such that only
finitely many dwarfs make a mistake. They can fix a strategy in advance, before getting their hats, but they cannot communicate after getting their hats.
Find a winning strategy for dwarfs.
\emph{Remark:} Problems \ref{zad:02-03}, \ref{zad:02-04} and \ref{zad:02-05} serve as a preparation for the Problem \ref{zad:02-06}.
}
{
Hint: consider the following relation on infinite 0-1 sequences $w \sim w'$ iff they differ on finitely many places.
We can treat a hat configuration as an infinite 0-1 sequence. Relation $\sim$ is an equivalence relation.
A strategy is to fix one element in every equivalence class. Every dwarf can easily recognize what is the equivalence
class. Then every one says the corresponding number of this one distinguished element in the equivalence class.
Only finitely many dwarfs will make a mistake.
}



\exercise{zad:02-04}{
Show that there is  a function $\xor: \{0, 1\}^\omega \to \{0, 1\}$, such that changing one bit of an argument always changes the result. (The solution uses the axiom of choice.)
}
{
In every equivalence class $C$ of $\sim$ choose one element $v_C$ and put $\xor(v_C) = 0$.
For every $v \in C$ put $\xor(v) = \fxor(v \otimes v_C)$, where $\otimes$ is the standard bit xor
and $\fxor$ outputs $0$ iff an argument has even number of $1$ (argument of $\fxor$ has to have
a finite number of ones). It is easy to verify that $\xor$ is indeed an infinite xor.
}



\exercise{zad:02-05}{
Consider the following two player game, called Chomp. There is a rectangular chocolate in a shape of $n\times k$ grid.
The right upper corner piece is rotten. Players move in an alternating manner, the first one moves first. 
Any player in his move picks square of the chocolate that is not yet eaten, and eats all pieces that are to
the left and to the bottom from the picked piece.
The player who eats the rotten piece loses. Determine who has a winning strategy.
}
{
The aim of this exercise is to present strategy stealing argument on a simple example.
We show that first player has a winning strategy for all $n, k$ (beside the trivial case $n = k = 1$)
without finding this strategy. As the chocolate game is a  game with a finite game tree, so  we know that it is determined by Exercise~\ref{zad:quasipol-reach}.
So for every position one of the players has a winning strategy. Fix $n, k \in \N$ such that $n > 1$ or $k > 1$.
Assume towards contradiction that second player has a winning strategy from the initial position.
The initial position is losing (as first player will move from it).
Consider a move of first player, which eats just one piece, the leftmost and bottommost.
The new position has to be winning. Then second player moves, eating some rectangle $n_1 \times k_1$
at the leftmost and bottommost corner. Notice that this is the only move second player can perform from this position.
This position has to be losing, as we assumed that second player used a winning strategy.
However, this position can be  already reached from from the initial position by a single move of first player.
This means that actually initial position was winning, as the owner can choose a move, which leads to a losing position.
Contradiction. Thus indeed it has to be the first player who has a winning strategy from the initial position.
}


\exercise{zad:02-06}{
Show a game that is  not determined.
}
{
Hint 1: use an infinite xor.
Hint 2: the game is as follows. There are two players, they construct an infinite 0-1 sequence.
Player One wins if $\xor$ of constructed element is $1$, otherwise Zero wins.
Zero and One construct this infinite sequence $w$ by delivering in an alternating manner a finite 0-1 sequences
and appending it to the currently constructed prefix of $w$.
Assume that Zero starts.

Observe now that this game is not determined. We will show that no player has a winning strategy.
Assume first that One has a winning strategy $\sigma$. We will show how Zero can sometimes win against this
strategy. Consider two plays $P_1$ and $P_2$.
Let Zero play $0$ in play $P_1$ and the response of One in $P_1$ be $v_0$.
Then Zero plays $1 v_0$ in $P_2$ and response of One in $P_2$ is some $v_1$.
Then Zero plays $v_1$ in $P_1$ and the response of One in $P_1$ is some $v_2$.
Then Zero plays $v_2$ in $P_2$ and the response of One in $P_2$ is some $v_3$.
In that way Zero copies responses of One to another play. Then in $P_1$ the constructed play
will be of the form $0 v_1 v_2 v_3 \cdots \in \{0, 1\}^\omega$ and in $P_2$ it will be of the
form $1 v_1 v_2 v_3 \cdots \in \{0, 1\}^\omega$. So the different player will win these plays, contradiction
with the assumption that $\sigma$ is a winning strategy for One.

Similarly we can get a contradiction with the assumption that Zero has a winning strategy.
Zero starts with some $v$ both in $P_1$ and in $P_2$. Then One plays $0$ in $P_1$, the response of Zero is $v_1$
and One plays $1 v_1$ in $P_2$. Then he copies Zero's responses as before and the results of these plays will be different
in $P_1$ and $P_2$. This leads to the contradiction.

So indeed this \xor-game is not determined.
}



\exercise{zad:02-15}{
Consider the following \emph{bisimilarity game} played on a finite game graph with vertices $V$ equipped with a function $\rank: V \to \N$.
Two players, \emph{Spoiler} and \emph{Duplicator} start from a position $(u, v) \in V \times V$.
The play proceeds in rounds.
If at the beginning of a round $\rank(u) \neq \rank(v)$ or $u$ and $v$ belong to different players then
Spoiler immediately wins. Otherwise Spoiler makes a move to $(u', v)$ or $(u, v')$ such that
$u \tran{} u'$ or $v \tran{} v'$, respectively. Then Duplicator makes a move to $(u', v')$ such that
$v \tran{} v'$ or $u \tran{} u'$, respectively. Next round starts from $(u', v')$. If play continues infinitely long then Duplicator wins.
Show that if Duplicator has a winning strategy from position $(u, v)$ then the same player has a winning strategy in the parity game
starting from $u$ and in the parity game starting in $v$.
}
{
Player $i$ will keep an invariant that vertices $u^k$ and $v^k$ (after $k$ rounds in both plays)
are bisimilar. This clearly suffices. Assume $u^k$ and $v^k$ are bisimilar. If both are vertices of player $i$
then player $i$ looks how he would move to $u^{k+1}$ and moves to $v^{k+1}$ such that it is the corresponding move.
Otherwise, player $i$ looks how his opponent moves to $v^{k+1}$ and thinks that he also
moves to $u^{k+1}$ by the corresponding move. Actually formally what we do we construct a strategy from $v$ winning for $i$
having the strategy from $u$ and the fact that $u$ and $v$ are bisimilar.
}



