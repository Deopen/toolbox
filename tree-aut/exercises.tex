% !TEX root = ../main.tex

\exercise{zad:05-01}{
The translation from \mso to automata in Theorem~\ref{thm:thatcher-wright} does an exponential blowup whenever it determinises the automaton, and therefore an upper bound on the running time is $n$-fold iteration of exponential, where $n$ is the size of the formula. Here is a matching lower bound.
Consider \mso on words, i.e.~there is a successor relation and unary predicates for the labels. Show that for every $n$, there is a formula of \mso (in fact, first-order logic is enough) which has size polynomial in $n$ and  is true in a unique word which has length
\begin{align*}
\underbrace{2^{2^{2^{2^{2^{\cdots ^{2^{2^{2^2}}}}}}}}}_{\text{$n$ times}}
\end{align*}
}
{
}


\exercise{zad:05-02}{
Show that the set $\Nat^*$ equipped with the prefix relation has decidable \mso theory.
}
{
}

\exercise{zad:05-03}{
Show that emptiness is polynomial time and universality is {\sc ExpTime}-complete for nondeterministic tree automata on finite trees.
\wojtek{Chcemy to? Ja mam watpliwosci.}
}
{
}

\exercise{zad:05-04}{
Show that emptiness for nondeterministic parity tree automata reduces in polynomial time to solving parity games.
}
{
}


\exercise{zad:05-06}{
Determine whether the following tree languages are regular:
\begin{enumerate}
  \item trees with an even number of nodes;
  \item trees with an even number of $a$-labelled nodes;
  \item trees over leaf alphabet $0, 1$ and internal alphabet $\vee, \wedge$ which evaluate to true
  when treated as boolean expressions;
  \item balanced trees (every leaf is at the same depth).
\end{enumerate}
}
{
In cases 1.-3. it is easy to show a nondeterministic automaton. Think that it goes bottom-up (it is usually a better perspective).
In 1. it counts number of nodes modulo $2$. Actually in 1. a tree which we consider is never accepted, because it always
have an odd number of nodes. In 2. it counts number of $a$-nodes modulo $2$. In 3. in remembers the boolean value
of the subtree.
In the case 4. language is not regular. It is easy to see. Consider the deterministic bottom-up automaton. Let $q_k$
be a state assigned to a complete binary tree of depth $k$. Let our automaton have $n$ states. Then by pigeonhole
principle some two among the trees $q_1, \ldots, q_{n+1}$ have the same state, say $q_i$ and $q_j$.
Then tree $a(q_i, q_i)$ and tree $a(q_i, q_j)$ will behave the same with respect to this automaton, but they shouldn't:
the first one is in the language, while the second one not.
}


\exercise{zad:05-05}{
Determine which of the following four variants of tree automata: deterministic / nondeterministic, top-down / bottom-up tree
automata are equivalent.
}
{
It is convenient to thing about the run of automaton a bit differently than before (in finite word case). Before we were usually
thinking that automaton is processing a word from left to right and assigns to every edge (between letters) a state.
Now it is better to think more declarative. Think that we label all the edges simultaneously and this labeling is correct
if it is consistent with transition relation. Then we easily see that nondeterministic top-down and bottom-up automata
has the some expressivity, as they actually have the same declarative definition.

We will now show that deterministic bottom-up variant is equivalently expressive, but deterministic top-down
variant is weaker. For focus on deterministic bottom-up variant.

We say that automaton is bottom-up deterministic if for every $p_1, p_2 \in Q$ and $a \in A$ there exists at most
one $p \in Q$ such that $(p, a, p_1, p_2)$ is a transition. We will show that for every nondeterministic automaton there
exists an equivalent bottom-up deterministic automaton. We just apply a subset construction bottom-up. A new state
will be the set of old states. An edge will be in the new state $S \subseteq Q$ if it can be in all old states $q \in S$
(there exists a labeling). One can easily see that this information can be updated bottom-up deterministically. A set of states
is final iff it contains at least one old final state.

Now let us show that deterministic top-down variant is weaker. We will show that it cannot recognize the language:
there exists an a-labeled node in the tree. This language can be easily recognize by a nondeterministic variant.
Assume that there is some deterministic top-down automaton $\A$ recognizing this language with initial state $q_0 \in Q$.
There is some transition $(q_0, b, q_L, q_R)$. If there is an $a$ in the left tree, but no $a$ in the right tree $\A$ should
reach final states everywhere, so there is an accepting run from $q_R$ on the right subtree even if there is no $a$ there.
Similarly there is an accepting run from $q_L$ on the left subtree even if there is no $a$ there. So $\A$ can accept also
trees such that there is no $a$ anywhere in the tree.
}





\exercise{zad:05-07}{
Define the \emph{yield} of a tree to be the word composed from labels of its leaves written in infix order.
Show that for every $L \subseteq \Sigma^*$ the following are equivalent
\begin{enumerate}
  \item $L$ is context-free;
  \item $L$ is the set of yields of some regular tree language.
\end{enumerate}

}
{
First implication from 1. to 2. Just consider a grammar in Chomsky normal form for $L$ and the regular language
of all its derivation trees. We can easily see that yield of a derivation is the derived word. So indeed the set of yields
of the regular language of derivations is $L$.

Implication from 2. to 1. is also not much harder. Just build a context-free grammar in Chomsky normal form from our regular
tree language. For every transition $(p, a, q, r)$ make a rule $X_p \tran{} X_q X_r$ in the grammar
and for every $(p, a, q, r)$, where $q$ and $r$ are accepting make a rule $X_p \tran{} a$ in the grammar.
Then the language of the grammar is exactly the set of yields of our regular tree language.
}




\exercise{zad:05-08}{
Show that deterministic top-down tree automata cannot recognize the language ''some node has label $a$''.
}
{

}




\exercise{zad:05-09}{
Show that the  language of words of even length is definable in \mso.
}
{
We will use sets $S$ and $T$ to mark odd and even positions, respectively.
We will also use macros $\first(x)$ defined as $\forall_{y \in X} x \leq y$,
$\last(x)$ defined as $\forall_{y \in X} x \geq y$ and $\nextpos(x, y)$ defines as
$(x \leq y) \wedge (\forall_{z \in X} \neg (x < z \wedge z < y))$.
The whole formula looks as follows
\[
\exists_{S, T \subseteq X} (\forall_{x \in X} x \in S \vee x \in T) \, \wedge 
\]
\[
(\forall_{x \in X} \neg (x \in S) \vee \neg (x \in T)) \, \wedge
\]
\[
(\forall_{x \in X} (\first(x) \Rightarrow x \in S) \wedge (\last(x) \Rightarrow x \in T)) \, \wedge
\]
\[
(\forall_{x, y \in X} (\nextpos(x, y) \Rightarrow (x \in S \iff y \in T))).
\]
x
}








\exercise{zad:05-12}{
Show that the following languages of infinite trees are regular (accepted by some nondeterministic automaton):
\begin{enumerate}
  \item on every path, the sequence of labels belongs to a given $\omega$-regular language $L$;
  \item some node has label $a$;
  \item in every subtree some node has label $a$.
\end{enumerate}
}
{
We construct automata as follows.
\begin{enumerate}
  \item We take a deterministic parity automaton for $L$ and on every path
  we use this automaton. Note that it is important that this automaton is deterministic, as it should behave
  the same on the prefix of two paths, which agree on some (finite) prefix.
  \item This one is simple, we just nondeterministically guess where is the letter $a$.
  State $q$ has to send into one child state $q$ (still searching for $a$) and into one child state $q'$ (accepting forever).
  \item This one is harder. Let assume wlog. that $\Sigma = \{a, b\}$. The automaton is as follows.
  It has two states: accepting $q_a$ and not accepting $q_r$. We have transitions $(q, a, q_a, q_a)$ for $q \in \{q_a, q_r\}$
  and transitions $(q, b, q_a, q_r)$, $(q, b, q_r, q_a)$ for $q \in \{q_a, q_r\}$. In other words if we see letter $a$ we send
  accepting states into both child and otherwise only to one child. Clearly if there is a subtree in which there is no letter $a$
  then for every run (so labeling) in this subtree there is an infinite path without accepting state.
  Indeed, we just always go down into the child, where the state $q_r$ was sent. Now we show that for every tree
  such that in every subtree there is a letter $a$ there exists an accepting run. We construct it. For every node let us choose
  some its descendant, which is labelled by $a$. Say for example that it is a shallowest descendant which is leftmost among
  the shallowest. Then if for a node $u$ descendant $v$ is chosen that for a node $u'$, a child of $u$, which is an ancestor of $v$
  also descendant $v$ is chosen. Then we construct a run: for every node the edge going down in the direction of chosen
  descendant is labelled by $q_r$ and the other one is labelled by $q_a$. This is really an accepting run. On every path either
  we follow the path to chosen descendant and after a finite time we hit letter $a$ and thus $q_a$ or we deviate from the path
  to chosen descendant and then we immediately have state $q_a$. Thus on every path we always have a finite time till
  the state $q_a$, so all the paths are accepted. Clearly acceptance and nonacceptance of
  states $q_a$ and $q_r$ can be implemented on ranks.
\end{enumerate}
}



\exercise{zad:05-13}{
In Existential Second Order Logic ($\eso$) one can write $\exists_{R_1, \ldots, R_n} \phi$, where
$R_i$ are any relations (possibly of arity greater than 1) and $\phi$ is a first order sentence (which of course may use $R_i$).
Show that the language of words of composite (non-prime) length is expressible in $\eso$.
}
{
We guess the relation $+k$ such that length of the word is $kn$.
We also guess the set of positions $0, k, 2k, (n-1)k$.
It is easy to verify, that our relation $R$ is of the form $+k$ for some $k$, we just
have to check that $+1(+k(x)) = +k(+1(x))$ for every $x$ ($+1$ is easy to implement using order).
Then we check that the set is indeed of the postulated form, the first position is in the set
and the last position in the set $-1$ and $+k$ is the last position in the word.

There are definitely another ways of solving this exercise.

This exercise is the special case of the more general fact (Fagin's theorem), which we will (maybe) show later.
}





\exercise{zad:05-18}{
Consider the following game. There are two players \emph{Insider} and \emph{Outsider}. They choose in an alternating manner
bits: $0$ or $1$ and create in that way an $\omega$-word $w$. If $w$ belongs to a given regular language
$W \subseteq \{0, 1\}^\omega$ then Insider wins a play, otherwise Outsider wins. Show that it is decidable to check which player
has a winning strategy in that game.
\emph{Remark:} use MSO logic.
}
{
We can express in MSO on trees that player Insider wins.
Let $W$ be represented by a formula $\psi$. We say that there exists a set $S$ of nodes in the tree (being
a set of infinite paths) such that
\begin{itemize}
  \item it contains the root
  \item it is really the set of infinite paths, no finite path there
  \item in every node $v$ belonging to Outsider (so on the even depth) in $S$ all the children of $v$ belong to $S$
  \item for every set $T \subseteq S$ which is an infinite path set $T$ satisfies $\psi$.
\end{itemize}
Therefore it is decidable.
}
